%% For double-blind review submission, w/o CCS and ACM Reference (max submission space)
\documentclass[acmsmall,review,anonymous]{acmart}\settopmatter{printfolios=true,printccs=false,printacmref=false}
%% For double-blind review submission, w/ CCS and ACM Reference
%\documentclass[acmsmall,review,anonymous]{acmart}\settopmatter{printfolios=true}
%% For single-blind review submission, w/o CCS and ACM Reference (max submission space)
%\documentclass[acmsmall,review]{acmart}\settopmatter{printfolios=true,printccs=false,printacmref=false}
%% For single-blind review submission, w/ CCS and ACM Reference
%\documentclass[acmsmall,review]{acmart}\settopmatter{printfolios=true}
%% For final camera-ready submission, w/ required CCS and ACM Reference
%\documentclass[acmsmall]{acmart}\settopmatter{}


%% Journal information
%% Supplied to authors by publisher for camera-ready submission;
%% use defaults for review submission.
\acmJournal{PACMPL}
\acmVolume{1}
\acmNumber{CONF} % CONF = POPL or ICFP or OOPSLA
\acmArticle{1}
\acmYear{2020}
\acmMonth{1}
\acmDOI{} % \acmDOI{10.1145/nnnnnnn.nnnnnnn}
\startPage{1}

%% Copyright information
%% Supplied to authors (based on authors' rights management selection;
%% see authors.acm.org) by publisher for camera-ready submission;
%% use 'none' for review submission.
\setcopyright{none}
%\setcopyright{acmcopyright}
%\setcopyright{acmlicensed}
%\setcopyright{rightsretained}
%\copyrightyear{2018}           %% If different from \acmYear

%% Bibliography style
\bibliographystyle{ACM-Reference-Format}
%% Citation style
%% Note: author/year citations are required for papers published as an
%% issue of PACMPL.
\citestyle{acmauthoryear}   %% For author/year citations


%%%%%%%%%%%%%%%%%%%%%%%%%%%%%%%%%%%%%%%%%%%%%%%%%%%%%%%%%%%%%%%%%%%%%%
%% Note: Authors migrating a paper from PACMPL format to traditional
%% SIGPLAN proceedings format must update the '\documentclass' and
%% topmatter commands above; see 'acmart-sigplanproc-template.tex'.
%%%%%%%%%%%%%%%%%%%%%%%%%%%%%%%%%%%%%%%%%%%%%%%%%%%%%%%%%%%%%%%%%%%%%%


%% Some recommended packages.
\usepackage{booktabs}   %% For formal tables:
                        %% http://ctan.org/pkg/booktabs
\usepackage{subcaption} %% For complex figures with subfigures/subcaptions
                        %% http://ctan.org/pkg/subcaption
\usepackage{stmaryrd}
\usepackage{todonotes}
\usepackage{amsthm}
\usepackage{amsthm}
\usepackage{amsmath}
\usepackage{amssymb}
\usepackage{stmaryrd}
\usepackage{semantic}
\usepackage{hyperref}
                        
\newtheorem{theorem}{Theorem}[]
\newtheorem{proposition}{Proposition}[]
\newtheorem{corollary}{Corollary}[]
\newtheorem{definition}{Definition}

\newcommand{\GTLC}{\texttt{GTLC+}}
\newcommand{\figref}[1]{Fig.~\ref{#1}}
\newcommand{\stxrule}[3]{#1 & ::= & #3 & \text{#2}\\}
\newcommand{\stxrulecont}[1]{& | & #1 & \\}
\newcommand{\funrule}[3]{#1 &=& #2 & #3\\}
\newcommand{\redrule}[3]{#1 & \longrightarrow_B & #2 & #3\\}
\newcommand{\comprule}[4]{#1 & \fatsemi^\ell & #2 & = & #3 & #4 \\}
\newcommand{\plus}[0]{+}
\newcommand{\judgetype}[3]{#1 \vdash #2 : #3}
\newcommand{\lazyUD}{$\mathtt{Lazy UD}$}
\newcommand{\lazyD}{$\mathtt{Lazy D}$}
\newcommand{\eager}{$\mathtt{Eager}$}
\newcommand{\hyperCoercionI}[0]{\mathtt{id\star}}
\newcommand{\hyperCoercionC}[3]{#1 \overset{#2}{\curvearrowright} #3}
\newcommand{\sOOinspect}[3]{\mathtt{Eval} \; #1 \; #2 \; #3}
\newcommand{\sOOreturn}[2]{\mathtt{Cont} \; #2 \; #1}
\newcommand{\sOOhalt}[1]{\mathtt{Halt} \; #1}
\newcommand{\TOOdyn}[0]{\star}
\newcommand{\TOOpre}[1]{#1}
\newcommand{\POOunit}[0]{\mathtt{Unit}}
\newcommand{\POOfun}[2]{#1 \shortrightarrow #2}
\newcommand{\POOprod}[2]{#1 \times #2}
\newcommand{\POOsum}[2]{#1 \plus #2}
\newcommand{\eOOvar}[1]{#1}
\newcommand{\eOOsole}[0]{\mathtt{unit}}
\newcommand{\eOOlam}[4]{\lambda^{#1\rightarrow{}#2}#3.#4}
\newcommand{\eOOapp}[2]{#1 \; #2}
\newcommand{\eOOcons}[2]{\mathtt{cons} \; #1 \; #2}
\newcommand{\eOOcar}[1]{\mathtt{fst} \; #1}
\newcommand{\eOOcdr}[1]{\mathtt{snd} \; #1}
\newcommand{\eOOinl}[1]{\mathtt{inl} \; #1}
\newcommand{\eOOinr}[1]{\mathtt{inr} \; #1}
\newcommand{\eOOcase}[3]{\mathtt{case} \; #1 \; #2 \; #3}
\newcommand{\eOOcast}[4]{#1 \langle \cOOcast{#2}{#3}{#4} \rangle}
\newcommand{\eOOblame}[1]{\mathtt{blame} \; #1}
\newcommand{\oOOinj}{\mathtt{dyn}}
\newcommand{\oOOsole}{\mathtt{unit}}
\newcommand{\oOOfun}{\mathtt{fun}}
\newcommand{\oOOcons}{\mathtt{cons}}
\newcommand{\oOOinl}{\mathtt{inl}}
\newcommand{\oOOinr}{\mathtt{inr}}
\newcommand{\oOOblame}[1]{\mathtt{blame} \; #1}
\newcommand{\cOOcast}[3]{#1 \overset{#2}{\Rightarrow} #3}
\newcommand{\vOOcast}[2]{#1\langle#2\rangle}
\newcommand{\vOOfun}[3]{\mathtt{fun} \; #1 \; #2 \; #3}
\newcommand{\vOOtt}[0]{\mathtt{unit}}
\newcommand{\vOOcons}[2]{\mathtt{cons}\;#1\;#2}
\newcommand{\vOOinl}[1]{\mathtt{inl}\;#1}
\newcommand{\vOOinr}[1]{\mathtt{inr}\;#1}
\newcommand{\rOOsucc}[1]{\mathtt{succ}\;#1}
\newcommand{\rOOfail}[1]{\mathtt{fail}\;#1}
\newcommand{\typingHC}[3]{#1 : #2 \Longrightarrow #3}
\newcommand{\hcvOOinj}[2]{\mathtt{inj} \; #2}
\newcommand{\hcvOOfun}[5]{\mathtt{fun} \; #1 \; #2 \; #3 \; #4 \; #5}
\newcommand{\hcvOOtt}[0]{\mathtt{unit}}
\newcommand{\hcvOOcons}[4]{\mathtt{cons}\;#1\;#2\;#3\;#4}
\newcommand{\hcvOOinl}[2]{\mathtt{inl}\;#1\;#2}
\newcommand{\hcvOOinr}[2]{\mathtt{inr}\;#1\;#2}
\newcommand{\hckOOmt}[0]{\mathtt{stop}}
\newcommand{\hckOOconsI}[3]{\mathtt{cons_1}\;#1\;#2\;#3}
\newcommand{\hckOOappII}[2]{\mathtt{app_2}\;#1\;#2}
\newcommand{\sidecond}[1]{\text{if}\;#1}

\begin{document}

%% Title information
\title{Blame and Coercion: Together Again for the Second Time}

%% Author information
%% Contents and number of authors suppressed with 'anonymous'.
%% Each author should be introduced by \author, followed by
%% \authornote (optional), \orcid (optional), \affiliation, and
%% \email.
%% An author may have multiple affiliations and/or emails; repeat the
%% appropriate command.
%% Many elements are not rendered, but should be provided for metadata
%% extraction tools.

%% Author with single affiliation.
\author{Kuang-Chen Lu}

\affiliation{
  \department{School of Informatics, Computing, and Engineering}              
  %% \department is recommended
  \institution{Indiana University}
  %% \institution is required
  \country{United States}
  %% \country is recommended
}
\email{kl13@iu.edu}          %% \email is recommended


\author{Jeremy Siek}

\orcid{nnnn-nnnn-nnnn-nnnn}             %% \orcid is optional
\affiliation{
  \position{Position2a}
  \department{Department2a}             %% \department is recommended
  \institution{Institution2a}           %% \institution is required
  \streetaddress{Street2a Address2a}
  \city{City2a}
  \state{State2a}
  \postcode{Post-Code2a}
  \country{Country2a}                   %% \country is recommended
}
\email{first2.last2@inst2a.com}         %% \email is recommended
\affiliation{
  \position{Position2b}
  \department{Department2b}             %% \department is recommended
  \institution{Institution2b}           %% \institution is required
  \streetaddress{Street3b Address2b}
  \city{City2b}
  \state{State2b}
  \postcode{Post-Code2b}
  \country{Country2b}                   %% \country is recommended
}
\email{first2.last2@inst2b.org}         %% \email is recommended


%% Abstract
%% Note: \begin{abstract}...\end{abstract} environment must come
%% before \maketitle command
\begin{abstract}
	Gradual Typing is great. Implementing gradually typed with blame tracking 
	and space efficiency is tricky. There exist two technique to do this: 
	coercion and threesome. Coercion is easy to understand, and easy enough to 
	implement, but difficult to reason about formally. Threesome is hard to 
	understand, easy to implement, and easy to reason about formally. We 
	propose hyper-coercion, which is easy to understand, as easy to implement 
	as coercion, and easy to reason about formally.
\end{abstract}


%% 2012 ACM Computing Classification System (CSS) concepts
%% Generate at 'http://dl.acm.org/ccs/ccs.cfm'.
\begin{CCSXML}
<ccs2012>
<concept>
<concept_id>10011007.10011006.10011008</concept_id>
<concept_desc>Software and its engineering~General programming languages</concept_desc>
<concept_significance>500</concept_significance>
</concept>
<concept>
<concept_id>10003456.10003457.10003521.10003525</concept_id>
<concept_desc>Social and professional topics~History of programming languages</concept_desc>
<concept_significance>300</concept_significance>
</concept>
</ccs2012>
\end{CCSXML}

\ccsdesc[500]{Software and its engineering~General programming languages}
\ccsdesc[300]{Social and professional topics~History of programming languages}
%% End of generated code


%% Keywords
%% comma separated list
\keywords{Gradual Typing, Blame, Coercion}
%% \keywords are mandatory in final camera-ready submission


%% \maketitle
%% Note: \maketitle command must come after title commands, author
%% commands, abstract environment, Computing Classification System
%% environment and commands, and keywords command.
\maketitle


\section{Introduction}

Around 2006, many work on integrating dynamic typing and static typing emerge, 
notably gradual types \cite{siek2006gradual} and hybrid types 
\cite{flanagan2006hybrid}. 
Later \citet{wadler2009well} introduces Blame Calculus, an intermediate 
language for gradual types and hybrid types. 
Blame Calculus throws a blame label when a runtime type-check fails.
These labels can include useful information on the failed cast, for example, 
its location in the source code.

Implementing gradually typed languages naively on top of blame calculus 
suffices sever space leak.
For example, translation to Blame Calculus can wrap some expressions with a 
cast, including those at tail position. Thus, at runtime, mutually 
tail-recursive functions in source language can accumulate casts at tail 
position. There are more examples in \citet{herman2010space}.

In 2007, \citet{herman2010space} proposes a solution to this problem by 
translating Blame Calculus to a variant of \citet{henglein1994dynamic}'s 
Coercion Calculus. Their key idea is to represent casts with coercions, 
which can be composed and normalized. Every coercion looks like a list of 
trees. For normal coercion, the length of the list is bounded by a small 
constant. And the depths of trees do not grow in composition and 
normalization. These two facts implies that coercion is a space-efficient cast 
representation. This Coercion Calculus, however, does not support blame 
tracking. 

After that, many efforts have been made to combine blame tracking and 
space efficiency. 

\citet{siek2009exploring} incorporate blame tracking into Coercion Calculus by 
decorating coercions with labels. They also propose that there are four blame 
strategies for their Coercion Calculi:
$ \{Lazy, Eager\} \times \{D, UD\} $. 
Eager strategies detect more potential type errors, 
but also blame more programs.
$ D $ and $ UD $ assign blame labels differently.
They prove partially that the \lazyUD{} Coercion Calculus is correct by showing 
that it simulates the Blame Calculus in \cite{wadler2009well}. It is unknown, 
however, whether the simulation also works the other direction.

\citet{siek2010threesomes} proposes another approach to combine blame 
tracking and space efficiency. Their solution is based on Threesome Calculus. 
Their key idea is to represent casts with threesomes, a novel cast 
representation.
%Threesomes can also be composed. Unlike coercions, there is no 
%separate normalization because every threesome is in normal form. 
They prove that their Threesome Calculus is correct by showing it bi-simulates
the \lazyUD{} Coercion Calculus in \citet{siek2009exploring} and the Blame 
Calculus in \citet{wadler2009well}.
They also prove that threesome is space-efficient as well.

\citet{siek2012interpretations} present the first \lazyD{} Blame Calculus and 
revised lazy Coercion Calculi.
They simplify the Coercion Calculi by only 
working with normal coercions (i.e. coercions in normal form). 
%Fortunately, coercions are trivially normal when initially constructed, and 
%their empowered compose function is conjectured to produces normal coercions 
%as well. 
They conjecture that their Coercion Calculi bi-simulate the corresponding Blame 
Calculi.

Following \citet{siek2010threesomes}, \citet{garcia2013calculating} discovers
Threesome Calculi for other blame strategies. After studying the relation 
between coercion and threesome,
he claims that normal coercion with labels is easy to understand but hard to 
implement and hard to reason about, and that 
threesome with labels, however, is easy to implement and easy to reason about 
but hard to understand. 
His claim is later affirmed by the group of people who develop threesome 
(\citet{siek2015blame}).
His threesome calculi are equivalent to the Coercion Calculi in 
\citet{siek2009exploring} because the former ones are derived from the latter 
ones.

\citet{siek2015blame} revisit the coercion-based approach. They revise the 
\lazyUD{} Coercion Calculus in \citet{siek2012interpretations}.
They prove that their calculus agrees with the \lazyUD{} Blame Calculus.

Last year, \citet{kuhlenschmidt2018efficient} present Grift, a compiler for a 
gradually typed language of the same name. 
This implementation is based on the \lazyD{} Coercion Calculus in 
\citet{siek2012interpretations}.
Their result suggests that normal coercion is easy enough to implement in a 
compiler. 

Recently, \citet{new2019gradual} show that \eager{} strategies are 
incompatible with $\eta$-equivalence of functions, which suggest that these 
strategies are not very ideal. 

So far, it seems that the coercion-based approach is the best way to combine 
space-efficiency and blame-tracking: normal coercion is claimed easy to 
understand \cite{garcia2013calculating}\cite{siek2015blame} and is shown 
easy enough to implement in a compiler \cite{kuhlenschmidt2018efficient}.
Normal coercion is still not satisfactory, however, in at least three aspects.
Firstly, as we mentioned above, it is hard to reason about. The major 
difficulty is from its non-structurally recursive composition.
For instance, many developers of Grift report convincing
their proof assistants that the composition terminates is tricky. 
Secondly, the definition of normal coercion lies unnecessarily on top 
of coercion: all coercions are normal when they are initially constructed, and 
\citet{siek2012interpretations} have shown that there exist functions that 
compose and produce normal coercions.

Perhaps unsurprisingly, threesome has a structurally recursive compose and a 
self-standing definition. Together with \citet{garcia2013calculating}'s 
work, they suggest that there might be a cast representation whose definition 
is self-standing, and whose composition is structurally recursive. 
Super-coercion introduced in \citet{garcia2013calculating} could be a promising 
candidate. However, its definition is a bit complicated. It uses 10 
constructors to deal with an elementary type system with only base types and 
function types. And four constructors are directly related to function types. 
Thus, super-coercion might not scale very well to more sophisticated type 
systems.

We present hyper-coercion, a cast representation whose composition is 
structurally recursive, and whose definition is self-standing. What's more, it 
should be at 
least as easy to implement and understand as normal coercions, because there is 
a clear connection between them, as we will show in Section 
\ref{sec:hyper-coercion}.

By comparing hyper-coercion with normal coercion, we notice another problem of 
the latter in memory representation. \dots 

Our hyper-coercion considers sum types and product types, which are not 
considered in all proofs above. Adding each of them requires us to add only one 
new constructor to a component of hyper-coercion. This suggests that 
hyper-coercion might scale better than super-coercion.

We prove formally that the \lazyD{} (resp. \lazyUD{}) hyper-coercion calculus 
bi-simulates the \lazyD{} (resp. \lazyUD{}) Blame Calculus. This is 
the first bi-simulation proof for the \lazyD{} Blame Calculus as far as we know.

The structure of this paper is as follows. 
Section \ref{sec:blame-calculus} reviews the state-of-art of $Lazy$ Blame 
Calculi. 
Section \ref{sec:coercion-calculus} reviews the state-of-art of $Lazy$ Coercion 
Calculi. 
In section \ref{sec:hyper-coercion} we present Hyper-coercion.
Section \ref{sec:conclude} concludes.

\section{Background: Blame Calculus} \label{sec:blame-calculus}

\begin{figure}
	Syntax
	\[
	\begin{array}{rclr}
	\stxrule{T}{types}{
		\star \mid{}
		P
	}
	\stxrule{P}{pre-types}{
		\POOunit \mid
		\POOfun{T_1}{T_2} \mid
		\POOprod{T_1}{T_2} \mid
		\POOsum{T_1}{T_2}
	}
	\stxrule{e}{terms}{
		\eOOvar{x} \mid{}
		\eOOsole{} \mid{}
		\eOOlam{T_1}{T_2}{x}{e} \mid
		\eOOapp{e_1}{e_2}
	}
	\stxrulecont{
		\eOOcons{e_1}{e_2} \mid
		\eOOcar{e} \mid
		\eOOcdr{e}
	}
	\stxrulecont{
		\eOOinl{e} \mid
		\eOOinr{e} \mid
		\eOOcase{e_1}{e_2}{e_3}
	}
	\stxrulecont{
		\eOOcast{e}{T_1}{l}{T_2} \mid
		\eOOblame{l}
	}
	\stxrule{o}{observations}{
		\oOOinj \mid
		\oOOsole \mid
		\oOOfun \mid
		\oOOcons \mid
		\oOOinl \mid
		\oOOinr \mid
		\oOOblame{l}
	}
	\end{array}
	\]
	
	Consistency
	\fbox{$ T_1 \sim T_2 $}
	\begin{gather*}
	\inference{}{
		\star \sim \star
	} \quad
	\inference{}{
		\star \sim P
	} \quad
	\inference{}{
		P \sim \star
	} \\
	\inference{}{
		\iota \sim \iota
	} \quad
	\inference{
		S_1 \sim S_2 &
		T_1 \sim T_2
	}{
		S_1 \rightarrow T_1 \sim S_2 \rightarrow T_2
	} \quad
	\inference{
		S_1 \sim S_2 &
		T_1 \sim T_2
	}{
		S_1 \times T_1 \sim S_2 \times T_2
	} \quad
	\inference{
		S_1 \sim S_2 &
		T_1 \sim T_2
	}{
		S_1 \plus T_1 \sim S_2 \plus T_2
	}
	\end{gather*}
	
	Shallow-consistency
	\fbox{$ T_1 \smile T_2 $}
	\begin{gather*}
	\inference{}{
		\star \smile \star
	} \quad
	\inference{}{
		\star \smile P
	} \quad
	\inference{}{
		P \smile \star
	} \\
	\inference{}{
		\iota \smile \iota
	} \quad
	\inference{}{
		T_{11} \rightarrow T_{12} \smile T_{21} \rightarrow T_{22}
	} \quad
	\inference{}{
		T_{11} \times T_{12} \smile T_{21} \times T_{22}
	} \quad
	\inference{}{
	T_{11} \plus T_1 \smile S_2 \plus T_2
	}
	\end{gather*}
	
	Term typing
	\fbox{$ \judgetype{\Gamma}{e}{T} $}
	\begin{gather*}
		\inference{
			S \sim T & \Gamma \vdash e : S 
		}{
			\judgetype{\Gamma}{\eOOcast{e}{T_1}{l}{T_2}}{T_2}
		} \quad
		\inference{
		}{
			\judgetype{\Gamma}{\eOOblame{l}}{T}
		}
	\end{gather*}
	
	\caption{Blame Calculus and its static semantics}
	\label{fig:blame-static}
\end{figure}

\figref{fig:blame-static} defines the syntax of blame calculus and its static 
semantics. It is little changed from previous definitions. 
The dynamic semantics of Blame Calculi depend on blame strategies, so we defer 
them to sub-sections.

Blame Calculus is based on Simply Typed Lambda Calculus with sum types and 
product types ($ \mathtt{STLC+} $). 
Let $ S,T $ range over types. A type is either the dynamic type $ \star $
(a.k.a. $ \mathtt{Dyn} $, $ \mathbb{?} $, or $ \mathtt{Unknown} $), 
or a pre-type. 
Let $ P,Q $ range over pre-types. Every pre-type is a type with a type 
constructor at the top. For simplicity, we only have one base type, $ \POOunit 
$. 
Other pre-types are functions, products, and sums.

%$ S \sim T $ reads ``$ S $ and $ T $ are consistent''.
%Two types are consistent if one of them is $ \star $, or they have the same 
%top-most type constructor and the corresponding sub-parts are consistent. 
%The intuition of $ S \sim T $ is that $ S $ and $ T $ have no conflict type 
%information. Consistency is reflexive and symmetric, but not transitive.

%One might be tempted to conclude that inconsistency is the root of all runtime 
%type errors (blames). In the setting of gradual typing, however, runtime 
%type-checking is usually gradual as well. 
$ S \smile T $ reads ``$ S $ and $ T $ are shallowly-consistent''. Two types 
are shallowly consistent if one of them is $ \star $, or they have the same 
top-most type constructor. Shallow-inconsistency is the root of all blames in 
all lazy strategies -- casting a value to a shallowly inconsistent type leads 
to a blame. Shallow-consistency is reflexive, symmetric, but 
not transitive.

Let $ e $ ranges over terms, including all terms from $ \mathtt{STLC+} $ and, 
in addition, casts and blames. Unlike $ \mathtt{STLC+} $, we annotate the 
co-domain of lambda abstractions explicitly. 
The end of the figure shows the typing rules for the additional terms.

Let $ o $ ranges over observations. They are what would be observed if a 
program terminates. Observations include all constructors and blames.

Now let's move to the dynamic semantics. 

\subsection{\lazyD{} Blame Calculus}

\begin{figure}
	Syntax
	\[
	\begin{array}{rclr}
	
	\stxrule{v}{values}{
		\vOOfun{\rho}{x}{b} \mid
		\vOOtt{} \mid
		\vOOcons{v_1}{v_2} \mid
		\vOOinl{v} \mid
		\vOOinr{v} \mid		
		\vOOcast{v}{c}
	}
	\stxrule{c}{casts}{
		\cOOcast{T_1}{l}{T_2}
	}
	\stxrule{r}{cast results}{
		\rOOsucc{v} \mid
		\rOOfail{l}
	}
	\stxrule{s}{states}{
		\sOOinspect{e}{\rho}{\kappa} \mid{}
		\sOOreturn{v}{\kappa} \mid{}
		\sOOhalt{o}
	}
		
	\stxrule{\kappa}{continuations}{
		\mathtt{stop} \mid{}
		\mathtt{cons_1} \; e \; \rho \; \kappa \mid{}
		\mathtt{cons_2} \; v \; \kappa \mid{}
		\mathtt{inl} \; \kappa \mid{}
		\mathtt{inr} \; \kappa
	}
	\stxrulecont{
		\mathtt{app_1} \; e \; \rho \; \kappa \mid{}
		\mathtt{app_2} \; v \; \kappa \mid{}
		\mathtt{car} \; \kappa \mid{}
		\mathtt{cdr} \; \kappa \mid
		\mathtt{case_1} \; e_1 \; e_2 \; \rho \; \kappa
	}
	\stxrulecont{	
		\mathtt{case_2} \; v   \; e   \; \rho \; \kappa \mid{}
		\mathtt{case_3} \; v_1 \; v_2 \; \rho \; \kappa \mid
		\langle c \rangle \kappa
	}
	\end{array}
	\]
	
	Value typing \fbox{$ v : T $}
	\begin{gather*}
	\dots \quad
	\inference{
		v : P &
		c : P \Longrightarrow T &
		P \smile T
	}{
		\vOOcast{v}{\cOOcast{P}{l}{T}} : T
	}
	\end{gather*}
	
%	Continuation typing \fbox{$ \kappa : T_1 \Longrightarrow T_2 $}
%	\begin{gather*}
%	\dots \quad
%	\inference{
%		c : T_1 \Longrightarrow T_2 &
%		\kappa : T_2 \Longrightarrow T_3
%	}{
%		\langle c \rangle \kappa : T_1 \Longrightarrow T_3
%	}
%	\end{gather*}
	
	Reduction \fbox{$ s \longrightarrow_{D,B} s $}
	\[
	\begin{array}{rclr}
	
	& \dots \\
	\redrule{
		\sOOinspect{\eOOcast{e}{T_1}{l}{T_2}}{\rho}{\kappa}
	}{
		\sOOinspect{e}{\rho}{\langle\cOOcast{T_1}{l}{T_2}\rangle\kappa}
	}{}
	\redrule{
		\sOOreturn{v_1}{(\mathtt{app_2} \; \vOOcast{v_2}{
				\cOOcast{\POOfun{T_1}{T_2}}{l}{\POOfun{T_3}{T_4}}
			} \; \kappa)}
	}{
		\sOOreturn{v_1}{
		\langle\cOOcast{T_3}{l}{T_1}\rangle
		(\mathtt{app_2} \; v_2 \; 
		\langle\cOOcast{T_2}{l}{T_4}\rangle \kappa)}
	}{}
	
%	\redrule{
%		\sOOreturn{v}{(
%			\mathtt{app_2} \;
%			(\vOOfun{\rho}{x}{e}) \;
%			\kappa
%		)}
%	}{\sOOinspect{e}{\rho{}[x := v]}{\kappa}}{}

	\redrule{
		\sOOreturn{
			\vOOcast{v}{\cOOcast{\POOprod{T_1}{T_2}}{l}{
					\POOprod{T_3}{T_4}}}
			}{(\mathtt{car} \; \kappa)}
	}{
		\sOOreturn{v}{(
			\mathtt{car} \;
			\langle \cOOcast{T_1}{l}{T_3} \rangle \kappa
			)}
	}{}

	\redrule{
		\sOOreturn{
			\vOOcast{v}{\cOOcast{\POOprod{T_1}{T_2}}{l}{
					\POOprod{T_3}{T_4}}}
		}{(\mathtt{cdr} \; \kappa)}
	}{
		\sOOreturn{v}{(
			\mathtt{cdr} \;
			\langle \cOOcast{T_2}{l}{T_4} \rangle \kappa
			)}
	}{}

	\redrule{
		\sOOreturn{v_3}{(
			\mathtt{case_3} \;
			v_1 \; v_2 \;
			\kappa
		)}
	}{case(v_1,
	\mathtt{app_2} \; v_2 \; \kappa,
	\mathtt{app_2} \; v_3 \; \kappa
	)}{}

	\redrule{
		\sOOreturn{v}{(
			\mathtt{cast} \; c \; \kappa
		)}
	}{
		\sOOreturn{u}{\kappa}
		\\ &
	}{
		\sidecond{applyCast(c,v) = \rOOsucc{u}}
	}

	\redrule{
		\sOOreturn{v}{(
			\mathtt{cast} \; c \; \kappa
			)}
	}{
		\sOOhalt{(\oOOblame{l})}
		\\ &
	}{
		\sidecond{applyCast(c,v) = \rOOfail{l}}
	}
	\end{array}
	\]	
	
	\fbox{$case(v, \kappa, \kappa) = s$}
	\[
	\begin{array}{rclr}
	\funrule{case(\vOOinl{v},\kappa_1,\kappa_2)}{
		\sOOreturn{v}{\kappa_1}
	}{}
	\funrule{case(\vOOinr{v},\kappa_1,\kappa_2)}{
		\sOOreturn{v}{\kappa_2}
	}{}
	\funrule{case(\vOOcast{v}{\cOOcast{\POOsum{T_{11}}{T_{12}}}{l}{\POOsum{T_{21}}{T_{22}}}},\kappa_1,\kappa_2)}{
		case(v,
		\langle \cOOcast{T_{11}}{l}{T_{21}} \rangle \kappa_1,
		\langle \cOOcast{T_{12}}{l}{T_{22}} \rangle \kappa_2)
	}{}
	\end{array}
	\]
	
	\fbox{$applyCast(c,v) = r$}
	\[
	\begin{array}{rclr}
	\funrule{
		applyCast(\cOOcast{\star}{l}{\star},v)
	}{
		\rOOsucc{v}
	}{}
	\funrule{
		applyCast(\cOOcast{\star}{l}{P_2},\vOOcast{v}{\cOOcast{P_1}{l'}{\star}})
	}{
		applyCast(\cOOcast{P_1}{l}{P_2},v)
	}{}
	\funrule{
		applyCast(\cOOcast{P}{l}{\star},v)
	}{
		\rOOsucc{\vOOcast{v}{\cOOcast{P}{l}{\star}}}
	}{}
	\funrule{
		applyCast(\cOOcast{P_1}{l}{P_2},v)
	}{
		\rOOsucc{\vOOcast{v}{\cOOcast{P_1}{l}{P_2}}}
	}{\sidecond{P_1 \smile P_2}}
	\funrule{
		applyCast(\cOOcast{P_1}{l}{P_2},v)
	}{
		\rOOfail{l}
	}{\sidecond{\neg P_1 \smile P_2}}
	
	\end{array}
	\]
	
	Transitive closure of reduction \fbox{$ s \longrightarrow_{D,B}^{*} s $}\\
	
	Evaluation \fbox{$ e \Downarrow_{D,B} o $}
	\[
	\inference{
		\sOOinspect{e}{\emptyset}{cont(\hckOOmt)} \longrightarrow_{B}^{*} 
		\sOOhalt{o}
	}{
		e \Downarrow_{D,B} o
	}
	\]
	
	\caption{CEK abstract machine for the \lazyD{} Blame Calculus}
	\label{machine-cekc}
\end{figure}

Fig.~\ref{machine-cekc} defines the dynamic semantcis of the \lazyD{} Blame 
Calculus in the style of CEK machines (\citet{felleisen1986control}). 
We choose CEK machine because to achieve space efficiency, we need to change 
continuations. Thus, abstract machines that include continuation in states is 
quite convenient for us. And CEK is one of the simplest machine of this kind.

Let $ c $ ranges over casts. A cast is a triple of a type, a label, and a type.
Let $ v $ ranges over values. A value is either a function, the 
element of unit, a pair, a left injection, a right injection, or a casted value.
The cast around a casted value is not arbitrary, as shown by value typing: the 
source type must be a pre-type, and the source and the target must be 
shallow-consistent.
Let $ r $ ranges over cast results. A cast result is either a success, which 
brings a value, or a failure, which brings a blame label.
Let $ s $ ranges over machine states. A state is either looking at an 
expression to decide what to do next, returning a value to a continuation, or 
halting with an observation.
Let $ \kappa $ ranges over continuations. $ \mathtt{mt} $ is the top 
continuation. The remaining continuations correspond to expressions. For 
example, $ (\mathtt{cons_1} \; e \; \rho \; \kappa) $ is the continuation where 
we are waiting for the value of the first argument to a $ \mathtt{cons} $. And 
the last continuation, $ \langle c \rangle \kappa $ is to cast the value before 
returning to $ \kappa $.

In the reduction rules, to evaluate a cast expression, the machine move the 
cast to the continuation and evaluate the inner expression. To apply a casted 
function, the machine firstly cast $ v_1 $, the operand, then apply the casted 
operand to $ v_2 $, the un-casted function, and finally cast the result of the 
function application. To take out the first (resp. second) part of a casted 
pair, the machine firstly take out the first (resp. second) part of $ v $, the 
un-casted pair, and cast the result. To case-split a sum, the machine looks 
deep inside the value to find out whether it is a left injection or a right 
one. Along the way, the machine accumulate the pending casts onto the 
continuations. To cast a value, the machine try to apply the cast to the value. 
If the cast succeeds, the machine returns the result to the next continuation. 
If the cast fails, the machine halts with the blame label.

Transitive closure of reduction and evaluation are standard.

\subsection{\lazyUD{} Blame Calculus [Jeremy]}


UNDER CONSTRUCTION


\section{Background: Coercion Calculus} \label{sec:coercion-calculus}

\todo[inline]{Which version of normal coercion should this section review?}

\subsection{\lazyD{} Coercion Calculus}

\subsection{\lazyUD{} Coercion Calculus}


\section{Space-efficient CEK Abstract Machine}

[move the definition of the generic machine here -Jeremy]


\section{Hyper-coercion} \label{sec:hyper-coercion}

\begin{figure}
	\[ 
	\begin{array}{lclr}
	\stxrule{c}{hyper-coercions}{
		\hyperCoercionI \mid{}
		\hyperCoercionC{h}{m}{t}
	}
	\stxrule{h}{heads}{
		\epsilon \mid{}
		?^l
	}
	\stxrule{m}{middles}{
		\POOunit \mid
		\POOfun{c_1}{c_2} \mid
		\POOprod{c_1}{c_2} \mid
		\POOsum{c_1}{c_2}
	}
	\stxrule{t}{tails}{
		\epsilon \mid{}
		! \mid{}
		\bot^l
	}
	\stxrule{\ell}{$ \mathtt{Maybe} \; l $}{
		\epsilon \mid{}
		l
	}
	\end{array}
	\]
	
	Shallow-consistency between middles
	\fbox{$ m \smile m $}
	
%	Coercion typing \fbox{$ c : T \Longrightarrow T $}
%	\begin{gather*}
%	\inference{}{\typingHC{\hyperCoercionI}{\TOOdyn}{\TOOdyn}}
%	\quad
%	\inference{
%		\typingHC{h}{T_1}{P_1} &
%		\typingHC{m}{P_1}{P_2} &
%		\typingHC{t}{P_2}{T_2}
%	}{
%		\typingHC{\hyperCoercionC{h}{m}{t}}{T_1}{T_2}
%	}
%	\end{gather*}
%	
%	Head typing \fbox{$ \typingHC{h}{T}{P} $}
%	\begin{gather*}
%	\inference{}{\typingHC{\epsilon}{P}{P}}
%	\quad
%	\inference{}{\typingHC{?^l}{\TOOdyn}{P}}
%	\end{gather*}
%	
%	Middle typing \fbox{$ \typingHC{m}{P}{P} $}
%	\begin{gather*}
%	\inference{}{\typingHC{\POOunit}{\POOunit}{\POOunit}}
%	\\
%	\inference{
%		\typingHC{c_1}{T_3}{T_1} &
%		\typingHC{c_2}{T_2}{T_4}
%	}{
%		\typingHC{\POOfun{c_1}{c_2}}{\POOfun{T_1}{T_2}}{\POOfun{T_3}{T_4}}
%	}
%	\\
%	\inference{
%		\typingHC{c_1}{T_1}{T_3} &
%		\typingHC{c_2}{T_2}{T_4}
%	}{
%		\typingHC{\POOprod{c_1}{c_2}}{\POOprod{T_1}{T_2}}{\POOprod{T_3}{T_4}}
%	}
%	\\
%	\inference{
%		\typingHC{c_1}{T_1}{T_3} &
%		\typingHC{c_2}{T_2}{T_4}
%	}{
%		\typingHC{\POOsum{c_1}{c_2}}{\POOsum{T_1}{T_2}}{\POOsum{T_3}{T_4}}
%	}
%	\end{gather*}
%	
%	Tail typing \fbox{$ \typingHC{t}{P}{T} $}
%	\begin{gather*}
%	\inference{}{\typingHC{\epsilon}{P}{P}} \quad
%	\inference{}{\typingHC{!}{P}{\TOOdyn}} \quad
%	\inference{}{\typingHC{\bot^l}{P}{T}} \quad
%	\end{gather*}
	
	\caption{Syntax of hyper-coercion}
	\label{fig:hyper-coercion}
\end{figure}

All definitions and results in this section are new.
\figref{fig:hyper-coercion} defines the syntax of hyper-coercions.
Terms, types, pre-types, blame labels, and observations are as before.

Let $ c $ ranges over hyper-coercions. We re-use the meta-variable $ c $ to 
stress that hyper-coercion is a cast representation. A hyper-coercion is either 
an identity cast between the dynamic types, or a complex including a head, a 
middle, and a tail. 
Let $ h $ ranges over heads. A head is either a no-op, or a projection 
decorated with a label, to which blame is allocated if the projection fails. 
Let $ m $ ranges over middles. There is a one-to-one 
correspondence between middles and type constructors. 
We generalize shallow-consistency to include middles in the natural way.
Let $ t $ ranges over tails. A tail is either a no-op, an injection, or a 
failure. 
Let $ \ell $ ranges over no-op and labels. It is useful when we introduce 
composition of hyper-coercions.

\subsection{\lazyD{} Hyper-coercion}

\begin{figure}
	Composition of hyper-coercions \fbox{$ c \fatsemi^\ell c = c $}
	\[ 
	\begin{array}{rclclr}
	
	\comprule{
		\hyperCoercionI
	}{
		\hyperCoercionI
	}{
		\hyperCoercionI
	}{}
	
	\comprule{
		\hyperCoercionI
	}{
		\hyperCoercionC{?^{l'}}{m}{t}
	}{
		\hyperCoercionC{?^{l'}}{m}{t}
	}{}
	
	\comprule{
		\hyperCoercionI
	}{
		\hyperCoercionC{\epsilon}{m}{t}
	}{
		\hyperCoercionC{?^{l}}{m}{t}
	}{\sidecond{\ell = l}}
	
	\comprule{
		\hyperCoercionC{h}{m}{\bot^{l'}}
	}{
		c
	}{
		\hyperCoercionC{h}{m}{\bot^{l'}}
	}{}
	
	\comprule{
		\hyperCoercionC{h}{m}{t}
	}{
		\hyperCoercionI
	}{
		\hyperCoercionC{h}{m}{!}
	}{
		\sidecond{\forall l. t \neq \bot^{l}}
	}
	
	\comprule{
		\hyperCoercionC{h}{m_1}{t_1}
	}{
		\hyperCoercionC{\epsilon}{m_2}{t_2}
	}{
		\hyperCoercionC{h}{m'}{t'}
	}{
		\sidecond{
			t_1 \neq \bot^{\centerdot} \; \text{and} \;
			m_1 \fatsemi^{\ell} (m_2, t_2) = (m', t')
		}
	}
	
	\comprule{
		\hyperCoercionC{h}{m_1}{t_1}
	}{
		\hyperCoercionC{?^{l'}}{m_2}{t_2}
	}{
		\hyperCoercionC{h}{m'}{t'}
	}{
		\sidecond{
		t_1 \neq \bot^{\centerdot} \; \text{and} \;
		m_1 \fatsemi^{l'} (m_2, t_2) = (m', t')
		}
	}
	\end{array}
	\]
	
	Composition of middles \fbox{$ m \fatsemi^\ell (m,t) = (m,t) $}
	\[ 
	\begin{array}{rclclr}
	\comprule{\POOunit}{(\POOunit,t)}{
		(\POOunit,t)
	}{}
	\comprule{\POOfun{c_1}{c_2}}{(\POOfun{c_3}{c_4},t)}{
		(\POOfun{c_3 \fatsemi^{\ell} c_1}{c_2 \fatsemi^\ell c_4},t)
	}{}
	\comprule{\POOprod{c_1}{c_2}}{(\POOprod{c_3}{c_4},t)}{
		(\POOprod{c_1 \fatsemi^\ell c_3}{c_2 \fatsemi^\ell c_4},t)
	}{}
	\comprule{\POOsum{c_1}{c_2}}{(\POOsum{c_3}{c_4},t)}{
		(\POOsum{c_1 \fatsemi^\ell c_3}{c_2 \fatsemi^\ell c_4},t)
	}{}
	\comprule{m_1}{(m_2,t)}{
		(m_1,\bot^l)
	}{
		\sidecond{
		\ell = l \; \text{and} \;
		\neg m_1 \smile m_2 
		}
	}
	\end{array}
	\]
	
	\fbox{$ seq(c,c) = c $}
	\[
	\begin{array}{rclr}
	\funrule{seq(c_1,c_2)}{
		c_1 \fatsemi^\epsilon c_2
	}{}
	\end{array}
	\]
	
	\fbox{$ id( P ) = m $}
	\[
	\begin{array}{rclr}
	\funrule{id(\POOunit)}{\POOunit}{}
	\funrule{id(\POOfun{T_1}{T_2})}{
		\POOfun{id(T_1)}{id(T_2)}
	}{}
	\funrule{id(\POOprod{T_1}{T_2})}{
		\POOprod{id(T_1)}{id(T_2)}
	}{}
	\funrule{id(\POOsum{T_1}{T_2})}{
		\POOsum{id(T_1)}{id(T_2)}
	}{}s
	\end{array}
	\]
	
	\fbox{$ id( T ) = c $}
	\[
	\begin{array}{rclr}
	\funrule{id(\star)}{
		\hyperCoercionI
	}{}
	\funrule{id(P)}{
		\hyperCoercionC{\epsilon}{id(P)}{\epsilon}
	}{}
	\end{array}
	\]
	
	\fbox{$ cast(T,l,T) = c$}
	\[
	\begin{array}{rclr}
	\funrule{cast(T_1,l,T_2)}{
		id(T_1) \fatsemi^l id(T_2)
	}{}
	\end{array}
	\]
	\caption{\lazyD{} Hyper-coercion}
	\label{fig:HC-D}
\end{figure}

Functions that construct \lazyD{} hyper-coercions are shown in 
\figref{fig:HC-D}. 

$ c_1 \fatsemi^\ell c_2 = c' $ composes two hyper-coercions. The target of $ 
c_1 $ and the source of $ c_2 $ might be different, in which case $ \ell $ must 
be a label.

$ m_1 \fatsemi^\ell (m_2,t) = (m',t') $ composes a middle with a 
pair of a middle and a tail. Similarly, the target of $ m_1 $ and the source of 
$ m_2 $ might be different, in which case $ \ell $ must be a label.

$ seq(c_1,c_2) $ composes two coercions where the target type of $ c_1 $ is 
equal to the source type of $ T_2 $.

$ id(T) $ constructs an identity coercion of $ T $. 

$ cast(T_1,l,T_2) $ constructs a coercion from a source type, a label, and a 
target type. 

\begin{proposition}[Hyper-coercions form a Monoid]
  For all $c : T_1 \Longrightarrow T_2$,
  $c_1 : T_1 \Longrightarrow T_2$,
  $c_2 : T_2 \Longrightarrow T_3$, and
  $c_3 : T_3 \Longrightarrow T_4$,
  \begin{enumerate}
  \item $seq(id(T_1),c) = c$,
  \item $seq(c,id(T_2)) = c$, and
  \item $seq(seq(c_1, c_2), c_3) = seq(c_1, seq(c_2, c_3))$.
  \end{enumerate}
\end{proposition}

To show that hyper-coercion is a correct cast representation, we introduce an 
hyper-coercion-based machine for the \lazyD\ blame calculus, and show that it 
bi-simulates the previous machine. 
\figref{machine-cekcc} defines the new machine.

\begin{figure}
	Syntax
	\[
	\begin{array}{rclr}
	
	\stxrule{v}{values}{
		\hcvOOtt \mid
		\hcvOOfun{c_1}{\rho}{x}{b}{c_2} \mid
		\hcvOOcons{v_1}{c_1}{v_2}{c_2}
	}
	\stxrulecont{
		\hcvOOinl{v}{c} \mid
		\hcvOOinr{v}{c} \mid
		\hcvOOinj{P}{v}
	}
	\stxrule{r}{cast results}{
		\rOOsucc{v} \mid
		\rOOfail{l}
	}
	\stxrule{s}{states}{
		\sOOinspect{e}{\rho}{\kappa} \mid{}
		\sOOreturn{v}{\kappa} \mid{}
		\sOOhalt{o}
	}
	\stxrule{\kappa}{continuation}{
		(c,k)
	}
	\stxrule{k}{pre-continuations}{
		\hckOOmt \mid{}
		\mathtt{cons_1} \; e \; \rho \; \kappa \mid{}
		\mathtt{cons_2} \; v \; \kappa \mid{}
		\mathtt{inl} \; \kappa \mid{}
		\mathtt{inr} \; \kappa
	}
	\stxrulecont{
		\mathtt{app_1} \; e \; \rho \; \kappa \mid{}
		\mathtt{app_2} \; v \; \kappa \mid{}
		\mathtt{car} \; \kappa \mid{}
		\mathtt{cdr} \; \kappa
	}
	\stxrulecont{
		\mathtt{case_1} \; e_1 \; e_2 \; \rho \; \kappa \mid
		\mathtt{case_2} \; v   \; e   \; \rho \; \kappa \mid{}
		\mathtt{case_3} \; v_1 \; v_2 \; \rho \; \kappa
	}
	\end{array}
	\]
	
	Build continuation \fbox{$ cont(k) = \kappa $}
	\[
	\begin{array}{rclc}
	\funrule{cont(k)}{(id(T_1),k)}{
		\sidecond{k : T_1 \Longrightarrow T_2}}
	\end{array}
	\]
	
	Extend continuation \fbox{$ ext(c,\kappa) = \kappa $}
	\[
	\begin{array}{rclc}
	\funrule{ext(c_1,(c_2,k))}{(seq(c_1,c_2),k)}{}
	\end{array}
	\]
	
	Reduction \fbox{$ s \longrightarrow_{D,P(C)} s $}
	\[
	\begin{array}{rclr}
	& \vdots \\
	\redrule{
		\sOOinspect{(\eOOlam{T_1}{T_2}{x}{e})}{\rho}{\kappa}
	}{
		\sOOreturn{(\hcvOOfun{id(T_1)}{\rho}{x}{e}{id(T_2)})}{\kappa}
	}{}
	\redrule{
		\sOOinspect{\eOOcast{e}{T_1}{l}{T_2}}{\rho}{\kappa}
	}{
		\sOOinspect{e}{\rho}{ext(cast(T_1,l,T_2),\kappa)}
	}{}
	\redrule{
		\sOOinspect{(\eOOcons{e_1}{e_2})}{\rho}{\kappa}
	}{
		\sOOinspect{e_1}{\rho}{cont(\hckOOconsI{e_2}{\rho}{\kappa})}
	}{}
	\redrule{
		\sOOreturn{(\hcvOOfun{c_1}{\rho}{x}{e}{c_2})}{(c,\hckOOappII{v}{\kappa})}
	}{
		\sOOinspect{e}{\rho[x:=v']}{ext(c_2,\kappa)}
	}{
		\\ & &
		\sidecond{applyCast(c_1,v) = \rOOsucc{v'}}
	}
	\redrule{
		\sOOreturn{(\hcvOOfun{c_1}{\rho}{x}{e}{c_2})}{(c,\hckOOappII{v}{\kappa})}
	}{
		\sOOhalt{(\oOOblame{l})}
	}{
		\\ & &
		\sidecond{applyCast(c_1,v) = \rOOfail{l}}
	}
	\end{array}
	\]
	
	Transitive closure of reduction \fbox{$ s \longrightarrow_{D,P(C)}^{*} s $}
	\[\dots\]
	
	Evaluation \fbox{$ e \Downarrow_{D,P(C)} o $}
	\[
	\inference{
		\sOOinspect{e}{\emptyset}{cont(\hckOOmt)} \longrightarrow_{P(C)}^{*} 
		\sOOhalt{o}
	}{
		e \Downarrow_{D,P(C)} o
	}
	\]
	
	\caption{An abstract machine that composes casts.}
	\label{machine-cekcc}
\end{figure}

Let $ v $ ranges over value. We push casts into values. For instance, function 
values now include two more components: a cast for argument and a cast for the 
output. The last kind of values is injection to $ \TOOdyn $. 
Cast results ($ r $) and machine states ($ s $) are as before. 
Let $ \kappa $ ranges over continuations and let $ k $ ranges over 
pre-continuations. 
A continuation is now a pair whose first part is a cast, and whose second part 
is a pre-continuation. 
Pre-continuation are like the continuations before.

We list a fraction of reduction rules due to space limitation.
When values are constructed, their hyper-coercion parts are filled with outputs 
of $ id $. For instance, when a function value is constructed, its first part 
and last part are initialized to $ id(T_1) $ and $ id(T_2) $ respectively.
When evaluating a cast expression, the current continuation is extended with a 
hyper-coercion constructed by $ cast $. $ ext $ composes the new hyper-coercion 
with the hyper-coercion on the top of the continuation.
When evaluating a compound expression, the machine firstly construct the new 
pre-continuation, then turn it to a continuation by adding an identity 
hyper-coercion at the top. For instance, when evaluating a \texttt{cons} 
expression, the machine firstly construct $ \hckOOconsI{e_2}{\rho}{\kappa} $, 
the new pre-continuation, then call $ cont $, which adds an identity cast to 
form a continuation. 
When a function call happens, the machine firstly cast the operand. If the 
casting succeeds, the machine then evaluate the body in the extended 
environment and the extended continuation. If the casting fails, the machine 
then halts with the blame label.

Transitive closure of reduction and evaluation are standard.

The machine depends on four functions that consumes and/or construct 
hyper-coercions: $ id(T) $, $ seq(c,c) $, $ cast(T,l,T) $, and $ applyCast(T,v) 
$. The first three are described above. We will show $ applyCast $ shortly.

Some reader might have noticed that the machine is representation-independent 
w.r.t cast. That being said, if some people want to use normal coercion instead 
of hyper-coercion, they can implement the normal coercion version of the four 
functions mentioned aboved and get another abstract machine. We capture the 
idea of ``abstract cast'' by the following definition.

\begin{definition}[Cast Representation]
	A cast representation has four interface functions:
	\begin{description}
		\item[$ id(T) $] constructs an identity cast
		\item[$ seq(c_1,c_2) $] composes two casts
		\item[$ cast(T_1,l,T_2) $] constructs a cast from $ T_1 $ to $ T_2 $
		\item[$ applyCast(c,v) $] applies a cast onto a value
	\end{description}
\end{definition}



\subsection{Correctness of \lazyD{} Hyper-coercions}



One of our major contributions is for all cast presentation, if it is proper, 
then it is correct.

\begin{definition}[Proper Cast Representation]
	A cast representation is proper if
	\begin{enumerate}
		\item If $ v : T $, then $ applyCast(id(T), v) = \mathtt{succ} \; v $
		\item If $ v : T_1 $,
		$ c_1 $ is from $ T_1 $ to $ T_2 $, and 
		$ c_2 $ is from $ T_2 $ to $ T_3 $,\\
		then $ applyCast(seq(c_1,c_2),v) = applyCast(c_1,v) >>= \lambda 
		v.applyCast(c_2,v) $
		\item If $ v : T_1 $ and $ \neg T_1 \smile T_2 $,
		then $ applyCast(cast(T_1, l, T_2),v) = \rOOfail{l} $
		\item If $ v : \star $, 
		then $ applyCast(cast(\TOOdyn,l,\TOOdyn),v) = \rOOsucc{v} $
		\item If $ v : P $,
		then $ applyCast(cast(\star,l,Q),\hcvOOinj{P}{v}) 
		= applyCast(cast(P,l,Q),v) $
		\item If $ v : P $,
		then $ applyCast(cast(P,l,\star),v) = \rOOsucc{(\hcvOOinj{P}{v})} $
		\item If $ v : \iota $,
		then $ applyCast(cast(\POOunit,l,\POOunit),v) = \rOOsucc{v} $
		\item $ 
		applyCast(cast(\POOfun{T_1}{T_2},l,\POOfun{T_3}{T_4}) ,
		\hcvOOfun{c_1}{\rho}{x}{b}{c_2}) \\
		= 
		\rOOsucc{(\hcvOOfun{seq(cast(T_3,l,T_1),c_1)}{\rho}{x}{b}{seq(c_2,cast(T_2,l,T_4))})}$
		
		\item $ applyCast(cast(\POOprod{T_1}{T_2},l,T_3 \times 
		T_4),\hcvOOcons{v_1}{c_1}{v_2}{c_2}) $ \\
		$ = 
		\rOOsucc{(\hcvOOcons{v_1}{seq(c_1,cast(T_1,l,T_3))}{v_2}{seq(c_2,cast(T_2,l,T_4))})}
		 $ 
		\item $ 
		applyCast(cast(\POOsum{T_1}{T_2},l,\POOsum{T_3}{T_4}),\hcvOOinl{v}{c})
		= \rOOsucc{(\hcvOOinl{v}{seq(c,cast(T_1,l,T_2))})} $
		\item $ 
		applyCast(cast(\POOsum{T_1}{T_2},l,\POOsum{T_3}{T_4}),\hcvOOinr{v}{c})
		= \rOOsucc{(\hcvOOinr{v}{seq(c,cast(T_3,l,T_4))})} $
	\end{enumerate}
\end{definition}

\begin{proposition}[proper cast represenations are correct]
  If $ \judgetype{\emptyset}{e}{T} $ and $ o : T $ and $ C $ is a proper cast 
  representation
  \[
  e \Downarrow_{D,B} o \; \text{if and only if} \; 
  e \Downarrow_{D,P(C)} o
  \]
\end{proposition}

Perhaps unsurprisingly, hyper-coercion is correct.

\begin{lemma}[\lazyD{} Hyper-coercion is a proper cast representation]
	See appendix.
\end{lemma}

\begin{theorem}[\lazyD{} hyper-coercion is correct]
	If $ \judgetype{\emptyset}{e}{T} $ and $ o : T $ 
	\[
	e \Downarrow_{D,B} o \; \text{if and only if} \; 
	e \Downarrow_{D,P(H)} o
	\]
\end{theorem}

\figref{hc-applyCast} defines $ applyCast $. Applying the identity cast for the 
dynamic type succeeds immediately. When applying a compound cast, we firstly 
apply the middle, then apply the tail. We denote by $ r \; >>= \; f $ to mean 
that if $ r $ is $ \rOOsucc{v} $, the result is $ f(v) $, otherwise the result 
is the failure.
$ applyMiddle(\ell,m,v) $ and $ applyTail(t,v) $ are straightforward.

\begin{figure}
	\fbox{$ applyCast(c,v) = r $}
	\[
	\begin{array}{rclr}
	\funrule{applyCast(\hyperCoercionI,\;v)}{\rOOsucc{v}}{}
	\funrule{applyCast(\hyperCoercionC{?^l}{m}{t},\;\hcvOOinj{P}{v})}{
		applyMiddle(l,m,v) \; >>= \; \lambda v. applyTail(t,v)
	}{}
	\funrule{applyCast(\hyperCoercionC{\epsilon}{m}{t},\;v)}{
		applyMiddle(\epsilon,m,v) \; >>= \; \lambda v. applyTail(t,v)
	}{}
	\end{array}
	\]
	
	\fbox{$ applyMiddle(\ell,m,v) = v $}
	\[
	\begin{array}{rclr}
	\funrule{applyMiddle(\ell,\POOunit,\hcvOOtt)}{\hcvOOtt}{}
	\funrule{applyMiddle(\ell,\POOfun{c_3}{c_4},\hcvOOfun{c_1}{\rho}{x}{e}{c_2})}{
		\hcvOOfun{(c_3 \fatsemi^\ell c_1)}{\rho}{x}{e}{(c_2 \fatsemi^\ell c_4)}
	}{}
	\funrule{applyMiddle(\ell,\POOprod{c_3}{c_4},\hcvOOcons{v_1}{c_1}{v_2}{c_2})}{
		\hcvOOcons{v_1}{(c_1 \fatsemi^\ell c_3)}{v_2}{(c_2 \fatsemi^\ell c_4)}
	}{}
	\funrule{applyMiddle(\ell,\POOsum{c_3}{c_4},\hcvOOinl{v}{c_1})}{
		\hcvOOinl{v}{(c_1 \fatsemi^\ell c_3)}
	}{}
	\funrule{applyMiddle(\ell,\POOsum{c_3}{c_4},\hcvOOinr{v}{c_2})}{
		\hcvOOinr{v}{(c_2 \fatsemi^\ell c_4)}
	}{}
	\funrule{applyMiddle(\ell,m,v)}{
		\rOOfail{l}
	}{
		\sidecond{\ell = l \; \text{and} \; \neg m \smile v}
	}
	\end{array}
	\]
	
	\fbox{$ applyTail(t,v) = r $}
	\[
	\begin{array}{rclr}
	\funrule{applyTail(\bot^l,v)}{\rOOfail{l}}{}
	\funrule{applyTail(\epsilon,v)}{\rOOsucc{v}}{}
	\funrule{applyTail(!,v)}{\rOOsucc{(\hcvOOinj{P}{v})}}{v : P}
	\end{array}
	\]
	\caption{Applying hyper-coercions to values}
	\label{hc-applyCast}
\end{figure}

\subsection{\lazyUD{} Hyper-coercion [Jeremy]}

UNDER CONSTRUCTION



\section{Conclusion} \label{sec:conclude}

%% Acknowledgments
\begin{acks}                            %% acks environment is optional
                                        %% contents suppressed with 'anonymous'
  %% Commands \grantsponsor{<sponsorID>}{<name>}{<url>} and
  %% \grantnum[<url>]{<sponsorID>}{<number>} should be used to
  %% acknowledge financial support and will be used by metadata
  %% extraction tools.
  This material is based upon work supported by the
  \grantsponsor{GS100000001}{National Science
    Foundation}{http://dx.doi.org/10.13039/100000001} under Grant
  No.~\grantnum{GS100000001}{nnnnnnn} and Grant
  No.~\grantnum{GS100000001}{mmmmmmm}.  Any opinions, findings, and
  conclusions or recommendations expressed in this material are those
  of the author and do not necessarily reflect the views of the
  National Science Foundation.
\end{acks}


%% Bibliography
\bibliography{bibfile}


%% Appendix
\appendix
\section{Appendix}

Text of appendix \ldots

\end{document}
