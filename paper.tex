%% For double-blind review submission, w/o CCS and ACM Reference (max submission space)
\documentclass[acmsmall,review,anonymous]{acmart}\settopmatter{printfolios=true,printccs=false,printacmref=false}
%% For double-blind review submission, w/ CCS and ACM Reference
%\documentclass[acmsmall,review,anonymous]{acmart}\settopmatter{printfolios=true}
%% For single-blind review submission, w/o CCS and ACM Reference (max submission space)
%\documentclass[acmsmall,review]{acmart}\settopmatter{printfolios=true,printccs=false,printacmref=false}
%% For single-blind review submission, w/ CCS and ACM Reference
%\documentclass[acmsmall,review]{acmart}\settopmatter{printfolios=true}
%% For final camera-ready submission, w/ required CCS and ACM Reference
%\documentclass[acmsmall]{acmart}\settopmatter{}


%% Journal information
%% Supplied to authors by publisher for camera-ready submission;
%% use defaults for review submission.
\acmJournal{PACMPL}
\acmVolume{1}
\acmNumber{CONF} % CONF = POPL or ICFP or OOPSLA
\acmArticle{1}
\acmYear{2020}
\acmMonth{1}
\acmDOI{} % \acmDOI{10.1145/nnnnnnn.nnnnnnn}
\startPage{1}

%% Copyright information
%% Supplied to authors (based on authors' rights management selection;
%% see authors.acm.org) by publisher for camera-ready submission;
%% use 'none' for review submission.
\setcopyright{none}
%\setcopyright{acmcopyright}
%\setcopyright{acmlicensed}
%\setcopyright{rightsretained}
%\copyrightyear{2018}           %% If different from \acmYear

%% Bibliography style
\bibliographystyle{ACM-Reference-Format}
%% Citation style
%% Note: author/year citations are required for papers published as an
%% issue of PACMPL.
\citestyle{acmauthoryear}   %% For author/year citations


%%%%%%%%%%%%%%%%%%%%%%%%%%%%%%%%%%%%%%%%%%%%%%%%%%%%%%%%%%%%%%%%%%%%%%
%% Note: Authors migrating a paper from PACMPL format to traditional
%% SIGPLAN proceedings format must update the '\documentclass' and
%% topmatter commands above; see 'acmart-sigplanproc-template.tex'.
%%%%%%%%%%%%%%%%%%%%%%%%%%%%%%%%%%%%%%%%%%%%%%%%%%%%%%%%%%%%%%%%%%%%%%


%% Some recommended packages.
\usepackage{booktabs}   %% For formal tables:
                        %% http://ctan.org/pkg/booktabs
\usepackage{subcaption} %% For complex figures with subfigures/subcaptions
                        %% http://ctan.org/pkg/subcaption
\usepackage{stmaryrd}
\usepackage{todonotes}
\usepackage{amsthm}
\usepackage{amsthm}
\usepackage{amsmath}
\usepackage{amssymb}
\usepackage{stmaryrd}
\usepackage{semantic}
\usepackage{hyperref}
                        
\newtheorem{theorem}{Theorem}[]
\newtheorem{proposition}{Proposition}[]
\newtheorem{corollary}{Corollary}[]
\newcommand{\figref}[1]{Fig.~\ref{#1}}
\newcommand{\stxrule}[3]{#1 & ::= & #3 & \text{#2}\\}
\newcommand{\stxrulecont}[1]{& | & #1 & \\}
\newcommand{\funrule}[3]{#1 & = & #2 & #3\\}
\newcommand{\comprule}[4]{#1 & \fatsemi^\ell & #2 & = & #3 & #4 \\}
\newcommand{\plus}[0]{+}
\newcommand{\judgetype}[3]{#1 \vdash #2 : #3}
\newcommand{\lazyUD}{$\mathtt{Lazy UD}$}
\newcommand{\lazyD}{$\mathtt{Lazy D}$}
\newcommand{\eager}{$\mathtt{Eager}$}
\newcommand{\hyperCoercionI}[0]{id\star}
\newcommand{\hyperCoercionC}[3]{#1 \overset{#2}{\curvearrowright} #3}

\begin{document}

%% Title information
\title{Blame and Coercion: Together Again for the Second Time}

%% Author information
%% Contents and number of authors suppressed with 'anonymous'.
%% Each author should be introduced by \author, followed by
%% \authornote (optional), \orcid (optional), \affiliation, and
%% \email.
%% An author may have multiple affiliations and/or emails; repeat the
%% appropriate command.
%% Many elements are not rendered, but should be provided for metadata
%% extraction tools.

%% Author with single affiliation.
\author{Kuang-Chen Lu}

\affiliation{
  \department{School of Informatics, Computing, and Engineering}              
  %% \department is recommended
  \institution{Indiana University}
  %% \institution is required
  \country{United States}
  %% \country is recommended
}
\email{kl13@iu.edu}          %% \email is recommended


\author{Jeremy Siek}

\orcid{nnnn-nnnn-nnnn-nnnn}             %% \orcid is optional
\affiliation{
  \position{Position2a}
  \department{Department2a}             %% \department is recommended
  \institution{Institution2a}           %% \institution is required
  \streetaddress{Street2a Address2a}
  \city{City2a}
  \state{State2a}
  \postcode{Post-Code2a}
  \country{Country2a}                   %% \country is recommended
}
\email{first2.last2@inst2a.com}         %% \email is recommended
\affiliation{
  \position{Position2b}
  \department{Department2b}             %% \department is recommended
  \institution{Institution2b}           %% \institution is required
  \streetaddress{Street3b Address2b}
  \city{City2b}
  \state{State2b}
  \postcode{Post-Code2b}
  \country{Country2b}                   %% \country is recommended
}
\email{first2.last2@inst2b.org}         %% \email is recommended


%% Abstract
%% Note: \begin{abstract}...\end{abstract} environment must come
%% before \maketitle command
\begin{abstract}
	Gradual Typing is great. Implementing gradually typed with blame tracking 
	and space efficiency is tricky. There exist two technique to do this: 
	coercion and threesome. Coercion is easy to understand, and easy enough to 
	implement, but difficult to reason about formally. Threesome is hard to 
	understand, easy to implement, and easy to reason about formally. We 
	propose hyper-coercion, which is easy to understand, as easy to implement 
	as coercion, and easy to reason about formally.
\end{abstract}


%% 2012 ACM Computing Classification System (CSS) concepts
%% Generate at 'http://dl.acm.org/ccs/ccs.cfm'.
\begin{CCSXML}
<ccs2012>
<concept>
<concept_id>10011007.10011006.10011008</concept_id>
<concept_desc>Software and its engineering~General programming languages</concept_desc>
<concept_significance>500</concept_significance>
</concept>
<concept>
<concept_id>10003456.10003457.10003521.10003525</concept_id>
<concept_desc>Social and professional topics~History of programming languages</concept_desc>
<concept_significance>300</concept_significance>
</concept>
</ccs2012>
\end{CCSXML}

\ccsdesc[500]{Software and its engineering~General programming languages}
\ccsdesc[300]{Social and professional topics~History of programming languages}
%% End of generated code


%% Keywords
%% comma separated list
\keywords{Gradual Typing, Blame, Coercion}
%% \keywords are mandatory in final camera-ready submission


%% \maketitle
%% Note: \maketitle command must come after title commands, author
%% commands, abstract environment, Computing Classification System
%% environment and commands, and keywords command.
\maketitle


\section{Introduction}

Around 2006, many work on integrating dynamic typing and static typing emerge, 
notably gradual types \cite{siek2006gradual} and hybrid types 
\cite{flanagan2006hybrid}. 
Later \citet{wadler2009well} introduces Blame Calculus, an intermediate 
language for gradual types and hybrid types. 
Blame Calculus throws a blame label when a runtime type-check fails.
Blame labels can include useful information on the failed cast, for example, 
its location in the source code.

Implementing gradually typed languages naively on top of blame calculus 
suffices sever space leak.
For example, translation to Blame Calculus can wrap some expressions with a 
cast, including those at tail position. Thus, at runtime, mutually 
tail-recursive functions in source language can accumulate casts at tail 
position. There are more examples in \citet{herman2010space}.

In 2007, \citet{herman2010space} proposes a solution to this problem by 
translating Blame Calculus to a variant of \citet{henglein1994dynamic}'s 
Coercion Calculus. Their key idea is to represent casts with coercions, 
which can be composed and normalized. Every coercion looks like a list of 
trees. For normalized coercion, the length of the list is bounded by a small 
constant. And the depths of trees do not growth in composition and 
normalization. These two facts implies that by representing casts with 
coercions, we can restore the space-efficiency of tail-recursive programs. This 
Coercion Calculus, however, doesn't include blame tracking. 

After that, many efforts have been made to combine blame tracking and 
space efficiency. 

\citet{siek2009exploring} incorporate blame tracking into Coercion Calculus by 
decorating coercions with labels. They also propose that there are four blame 
strategies for their Coercion Calculi:
$ \{Lazy, Eager\} \times \{D, UD\} $. 
Eager strategies detect more potential type errors, 
but also blame more programs.
$ D $ and $ UD $ assign blame labels differently.
Blame strategy is essentially the semantics of casts.
They prove that the \lazyUD{} Coercion Calculus simulates the Blame Calculus 
in \cite{wadler2009well}. But we don't know whether the simulation also works 
the other way, and what are the Blame counterparts of other Coercion Calculi.

\citet{siek2010threesomes} proposes another approach to combine blame 
tracking and space efficiency. Their solution is based on Threesome Calculus. 
Their key idea is to represent casts with threesomes, a novel cast 
representation. Threesomes, like coercions, can be composed. There is no 
separate normalization because every threesome is in normal form. 
They prove that their Threesome Calculus bi-simulates
the \lazyUD{} Coercion Calculus and the Blame Calculus.

\citet{siek2012interpretations} introduces a \lazyD{} Blame Calculus. They 
conjecture that it bi-simulates the \lazyD{} Coercion Calculus.

Following \citet{siek2010threesomes}, \citet{garcia2013calculating} 
implement all other blame strategies with Threesome Calculus. 
He claims that 
coercion with labels is easy to understand but hard to implement, 
and that 
threesome with labels, however, is easy to implement but hard to understand.
His claim is later affirmed by the group of people who develop threesome 
(\citet{siek2015blame}).
The connection 
between his Threesome Calculi and \citet{siek2009exploring}'s Coercion 
Calculi are established by the fact that the former are derived from the latter.
The connection between eager calculi and Blame Calculus, however, 
is still unclear.

\citet{siek2015blame} revisit the coercion-based approach. They simplify the 
\lazyUD{} Coercion Calculus by only working with normalized coercions.
Fortunately, coercions are trivially normalized when initially constructed, and 
their empowered compose function produces normalized coercions as well. Again 
\cite{siek2010threesomes}, they delegate all other blame strategies.

Last year, \citet{kuhlenschmidt2018efficient} present Grift, a space-efficient 
and blame-tracking compiler for a gradually typed language of the same name. 
This implementation is based on the \lazyD{} Coercion Calculus.
Their result suggests that implementing coercion is practical. They implement 
tuples, which is the first time gradual product types is considered. Their 
treatment on tuples, however, is shown incorrect 
\cite{Gradual-TypingType-based-casts}.

Recently, \citet{new2019gradual} show that \eager{} strategies are 
incompatible with $\eta$-equivalence of functions, which suggest that these 
strategies are not very ideal. 

So far, it seems that the coercion-based approach is the best way to combine 
space-efficiency and blame-tracking: coercion is claimed easy to 
understand \cite{garcia2013calculating}\cite{siek2015blame} and is shown 
easy enough to implement in a compiler \cite{kuhlenschmidt2018efficient}.
Coercion is not satisfactory, however, in at least two aspects.
Firstly, its compose function is not structurally recursive, which makes 
it difficult to study coercion formally.
Many developers of Grift report that it is tricky to convince their 
proof assistants that composition terminates. 
Secondly, it is desirable to understand the nature of ``normalized cast'', but 
the coercion-based approach capture this concept indirectly because normalized 
coercion is defined ny taking a subset of coercion.

Perhaps unsurprisingly, threesome has a structurally recursive compose. And its 
definition is self-standing. Together with \citet{garcia2013calculating}'s 
work, they suggest that there can be a cast representation whose definition is 
self-standing, and whose composition is structurally recursive. 
Super-coercion introduced in \citet{garcia2013calculating} could be a promising 
candidate. However, its definition is a bit complicated. It uses 10 
constructors to deal with an elementary type system with only base types and 
function types. And four constructors are directly related to function types. 
Thus, super-coercion might not scale very well to more sophisticated type 
systems.

Now we present one such cast representation: hyper-coercion. As we will show in 
Section \ref{sec:hyper-coercion}, there is a clear connection between it and 
the normalized coercion, so we are optimistic that understanding and 
implementing hyper-coercion should be as easy (or as hard).

Our hyper-coercion considers sum types and product types, which are not 
accounted in all proofs above. Adding each of them requires us to add only one 
new constructor to a component of hyper-coercion. This suggests that 
hyper-coercion might scale better than super-coercion.

We prove formally that the \lazyD{} (resp. \lazyUD{}) hyper-coercion calculus 
bi-simulate the \lazyD{} (resp. \lazyUD{}) Blame Calculus. This is possibly 
the first bi-simulation proof for the \lazyD{} Blame Calculus.

The structure of this paper is as follows. 
Section \ref{sec:blame-calculus} reviews the state-of-art of $Lazy$ Blame 
Calculi. 
In section \ref{sec:hyper-coercion} we present Hyper-coercion.
Section \ref{sec:conclude} concludes.

\section{Background: Blame Calculus} \label{sec:blame-calculus}

\begin{figure}
	\[
	\begin{array}{lclr}
	\stxrule{S,T}{types}{
		\star \mid{}
		P
	}
	\stxrule{P,Q}{pre-types}{
		\iota \mid{}
		T_1 \rightarrow T_2 \mid{}
		T_1 \times T_2 \mid{}
		T_1 \plus T_2
	}
%	\stxrule{x,y,z}{variables}{\dots}
%	\stxrule{l}{labels}{\dots}
%	\stxrule{k}{constants}{sole}
	\stxrule{e}{terms}{
		x \mid{}
		\mathtt{tt} \mid{}
		\lambda^{S\rightarrow{}T}x.t \mid{}
		e_1 \; e_2
	}
	\stxrulecont{
		\mathtt{cons}\; e_1 \; e_2 \mid{}
		\mathtt{car}\; e \mid{}
		\mathtt{cdr}\; t
	}
	\stxrulecont{
		\mathtt{inl} \; e \mid{}
		\mathtt{inr} \; e \mid{}
		\mathtt{case}\; e_1 \; e_2 \; e_3
	}
	\stxrulecont{
		\langle T \Leftarrow^l S \rangle t \mid{}
		\mathtt{blame} \; l
	}
	\stxrule{o}{observations}{
		\mathtt{tt} \mid{}
		\mathtt{fun}\mid{}
		\mathtt{cons}\mid{}
		\mathtt{inl}\mid{}
		\mathtt{inr}\mid{}
		\mathtt{blame}\; l
	}
	\end{array}
	\]
	
	\fbox{$ S \sim T $}
	\begin{gather*}
	\inference{}{
		\star \sim \star
	} \quad
	\inference{}{
		\star \sim P
	} \quad
	\inference{}{
		P \sim \star
	} \\
	\inference{}{
		\iota \sim \iota
	} \quad
	\inference{
		S_1 \sim S_2 &
		T_1 \sim T_2
	}{
		S_1 \rightarrow T_1 \sim S_2 \rightarrow T_2
	} \quad
	\inference{
		S_1 \sim S_2 &
		T_1 \sim T_2
	}{
		S_1 \times T_1 \sim S_2 \times T_2
	} \quad
	\inference{
		S_1 \sim S_2 &
		T_1 \sim T_2
	}{
		S_1 \plus T_1 \sim S_2 \plus T_2
	}
	\end{gather*}
	
	\fbox{$ S \smile T $}
	\begin{gather*}
	\inference{}{
		\star \smile \star
	} \quad
	\inference{}{
		\star \smile P
	} \quad
	\inference{}{
		P \smile \star
	} \\
	\inference{}{
		\iota \smile \iota
	} \quad
	\inference{}{
		S_1 \rightarrow T_1 \smile S_2 \rightarrow T_2
	}
	\inference{}{
	S_1 \times T_1 \smile S_2 \times T_2
	}
	\inference{}{
	S_1 \plus T_1 \smile S_2 \plus T_2
	}
	\end{gather*}
	
	\fbox{$ \Gamma \vdash e \vdash T $}
	\begin{gather*}
		\inference{
			S \sim T & \Gamma \vdash e : S 
		}{
			\judgetype{\Gamma}{\langle T \Leftarrow^l S \rangle t}{T}
		} \quad
		\inference{
		}{
			\judgetype{\Gamma}{\mathtt{blame} \; l}{T}
		}
	\end{gather*}
	
	\caption{Blame Calculus and its static semantics}
	\label{fig:blame-static}
\end{figure}

\figref{fig:blame-static} defines the syntax of blame calculus and its static 
semantics. It is little changed from previous definitions. 
The dynamic semantics of Blame Calculi depend on blame strategies, so we defer 
them to sub-sections.

Blame Calculus is based on Simply Typed Lambda Calculus with sum types and 
product types ($ \mathtt{STLC+} $). 
Let $ S,T $ range over types. A type is either the dynamic type $ \star $
(a.k.a. $ \mathtt{Dyn} $, $ \mathbb{?} $, or $ \mathtt{Unknown} $), 
or a pre-type. 
Let $ P,Q $ range over pre-types. Every pre-type is a type with a type 
constructor at the top. For simplicity, we only have one base type $ \iota $, 
the unit type. Other pre-types are functions, products, and sums.

$ S \sim T $ reads ``$ S $ and $ T $ are consistent''.
Two types are consistent if one of them is $ \star $, or they have the same 
top-most type constructor and the corresponding sub-parts are consistent. 
The intuition of $ S \sim T $ is that $ S $ and $ T $ have no conflict type 
information. Consistency is reflexive and symmetric, but not transitive.

One might be tempted to conclude that inconsistency is the root of all runtime 
type errors (blames). In the setting of gradual typing, however, runtime 
type-checking is usually gradual as well. 
$ S \smile T $ reads ``$ S $ and $ T $ are shallowly-consistent''. Two types 
are shallowly consistent if one of them is $ \star $, or they have the same 
top-most type constructor. Shallow-inconsistency is the root of all blames in 
all lazy strategies -- casting a value to a shallowly inconsistent type leads 
to a blame immediately. Shallow-consistency is also reflexive, symmetric, and 
not transitive.

Let $ e $ ranges over terms, including all terms from $ \mathtt{STLC+} $, 
casts, and blames. Unlike $ \mathtt{STLC+} $, we annotate the 
co-domain of lambda abstractions explicitly. 
The end of the figure shows the extra typing rules.

Let $ o $ ranges over observations. They are what would be observed if a 
program terminates. Observations include all constructors and blames.

Now let's move to the dynamic semantics. 

\subsection{$Lazy D$ Blame Calculus}

All existing proofs describe the semantics of \lazyD{} Blame Calculus in the 
style of reduction semantics. 
We instead do it with the CEK machine (\citet{felleisen1986control}). 
We choose CEK machine because to achieve space efficiency, we need to change 
continuations. Thus, abstract machines that include continuation in states is 
quite convenient for us. And CEK is one of the simplest machine of this kind.

\subsection{$Lazy UD$ Blame Calculus}

subtyping, reduction ...

\section{Background: Coercion Calculus}

\subsection{$Lazy D$ Coercion Calculus}

\subsection{$Lazy UD$ Coercion Calculus}

\section{Hyper-coercion} \label{sec:hyper-coercion}

\begin{figure}
	\[ 
	\begin{array}{lclr}
	\stxrule{c}{hyper-coercions}{
		\hyperCoercionI \mid{}
		\hyperCoercionC{h}{m}{t}
	}
	\stxrule{h}{heads}{
		\epsilon \mid{}
		?^l
	}
	\stxrule{m}{middles}{
		U \mid{}
		c_1 \rightarrow c_2 \mid{}
		c_1 \times c_2 \mid{}
		c_1 \plus c_2
	}
	\stxrule{t}{tails}{
		\epsilon \mid{}
		! \mid{}
		\bot^l
	}
	\stxrule{\ell}{$ \mathtt{Maybe} \; l $}{
		\epsilon \mid{}
		l
	}
	\end{array}
	\]
	
	\fbox{$ c \fatsemi^\ell c = c $}
	\[ 
	\begin{array}{rclclr}

	\comprule{
		\hyperCoercionI
	}{
		\hyperCoercionI
	}{
		\hyperCoercionI
	}{}

	\comprule{
		\hyperCoercionI
	}{
		\hyperCoercionC{?^{l'}}{b}{t}
	}{
		\hyperCoercionC{?^{l'}}{b}{t}
	}{}

	\comprule{
		\hyperCoercionI
	}{
		\hyperCoercionC{!}{b}{t}
	}{
		\hyperCoercionC{?^{l}}{b}{t}
	}{\ell = l}

	\comprule{
		\hyperCoercionC{h}{b}{\bot^{l'}}
	}{
		c
	}{
		\hyperCoercionC{h}{b}{\bot^{l'}}
	}{}

	\comprule{
		\hyperCoercionC{h}{b_1}{t_1}
	}{
		\hyperCoercionI
	}{
		\hyperCoercionC{h}{b_1}{!}
	}{
		t_1 \neq \bot^{\centerdot}
	}

	\comprule{
		\hyperCoercionC{h}{b_1}{t_1}
	}{
		\hyperCoercionC{\epsilon}{b_2}{t_2}
	}{
		\hyperCoercionC{h}{b'}{t'}
	}{
		t_1 \neq \bot^{\centerdot} \; \text{and} \;
		b_1 \fatsemi^{\ell} (b_2, t) = (b', t')
	}

	\comprule{
		\hyperCoercionC{h}{b_1}{t_1}
	}{
		\hyperCoercionC{?^{l'}}{b_2}{t_2}
	}{
		\hyperCoercionC{h}{b'}{t'}
	}{
		t_1 \neq \bot^{\centerdot} \; \text{and} \;
		b_1 \fatsemi^{l'} (b_2, t) = (b', t')
	}
	\end{array}
	\]
	
	\fbox{$ m \fatsemi^\ell (m,t) = (m,t) $}
	\[ 
	\begin{array}{rclclr}
	\comprule{U}{(U,t)}{
		(U,t)
	}{}
	\comprule{c_1 \rightarrow c_2}{(c_3 \rightarrow c_4,t)}{
		(c_3 \fatsemi^{l} c_1 \rightarrow c_2 \fatsemi^\ell c_4, t)
	}{}
	\comprule{c_1 \times c_2}{(c_3 \times c_4,t)}{
		(c_1 \fatsemi^{l} c_3 \times c_2 \fatsemi^\ell c_4, t)
	}{}
	\comprule{c_1 \plus c_2}{(c_3 \plus c_4,t)}{
		(c_1 \fatsemi^{l} c_3 \plus c_2 \fatsemi^\ell c_4, t)
	}{}
	\comprule{b_1}{(b_2,t)}{
		(b_1,\bot^l)
	}{
		\ell = l \; \text{and} \;
		\neg b_1 \smile b_2 
	}
	\end{array}
	\]
	
	\fbox{$ seq(c,c) = c $}
	\[
	\begin{array}{rclr}
	\funrule{seq(c_1,c_2)}{
		c_1 \fatsemi^\epsilon c_2
	}{cod(c_1) = dom(c_2)}
	\end{array}
	\]
	
	\fbox{$ id( P ) = m $}
	\[
	\begin{array}{rclr}
	\funrule{id(U)}{U}{}
	\funrule{id(T_1 \rightarrow T_2)}{
		id(T_1) \rightarrow id(T_2)
	}{}
	\funrule{id(T_1 \times T_2)}{
		id(T_1) \times id(T_2)
	}{}
	\funrule{id(T_1 \plus T_2)}{
		id(T_1) \plus id(T_2)
	}{}
	\end{array}
	\]
	
	\fbox{$ id( T ) = c $}
	\[
	\begin{array}{rclr}
	\funrule{id(\star)}{
		\hyperCoercionI
	}{}
	\funrule{id(P)}{
		\hyperCoercionC{\epsilon}{id(P)}{\epsilon}
	}{}
	\end{array}
	\]
	
	\fbox{$ cast(T,l,T) = c$}
	\[
	\begin{array}{rclr}
	\funrule{cast(T_1,l,T_2)}{
		id(T_1) \fatsemi^l id(T_2)
	}{}
	\end{array}
	\]
	

	\caption{$Lazy D$ Hyper-coercion}
	\label{fig:HC-D}
\end{figure}

\subsection{$Lazy D$ Hyper-coercion}

The syntax of \lazyD{} Hyper-coercion is shown in \figref{fig:HC-D}.

\begin{theorem}[\lazyD{} Hyper-coercion is a proper cast representation]
	\  \\
\begin{enumerate}
	\item If $ v : T $, then $ id(T) \; v = \mathtt{succ} \; v $
	\item If $ v : T_1 $,
	$ c_1 $ is a hyper-coercion from $ T_1 $ to $ T_2 $, and 
	$ c_2 $ is from $ T_2 $ to $ T_3 $,\\
	then $ seq(c_1,c_2) \; v = (c_1 \; v) >>= c_2 $
	\item If $ v : T_1 $ and $ \neg T_1 \smile T_2 $,
	then $ cast(T_1, l, T_2) \; v = \mathtt{fail} \; l $
	\item If $ v : \star $, 
	then $ cast(\star,l,\star) \; v = \mathtt{succ} \; v $
	\item If $ v : P $,
	then $ cast(\star,l,Q) \; (inj \; P \; v) = cast(P,l,Q) \; v $
	\item If $ v : P $,
	then $ cast(P,l,\star) \; v = \mathtt{succ} \; (inj \; P \; v) $
	\item If $ v : \iota $,
	then $ cast(\iota,l,\iota) \; v = \mathtt{succ} \; v $
	\item $ cast(S_1 \rightarrow T_1,l,S_2 \rightarrow T_2) \; (fun \; c_1 \; 
	E \; b \; c_2) $ \\
	$ = \mathtt{succ} \; fun \; seq(cast(S_2,l,S_1),c_1) \; E \; b 
	\; seq(c_2,cast(T_1,l,T_2)) $ 
	\item $ cast(S_1 \times T_1,l,S_2 \times T_2) \; (cons \; v_1 \; c_1 \; 
	v_2 \; c_2) $ \\
	$ = \mathtt{succ} \; (cons \; v_1 \; seq(c_1,cast(S_1,l,S_2)) \; v_2 \; 
	seq(c_2,cast(T_1,l,T_2))) $ 
	\item $ cast(S_1 \plus T_1,l,S_2 \plus T_2) \; (inl \; v \; c) = 
	\mathtt{succ} \; (inl \; v \; seq(c,cast(S_1,l,T_1)) ) $ 
	\item $ cast(S_1 \plus T_1,l,S_2 \plus T_2) \; (inr \; v \; c) = 
	\mathtt{succ} \; (inr \; v \; seq(c,cast(S_2,l,T_2)) ) $ 
\end{enumerate}
\end{theorem}

\begin{proposition}[Every proper cast represenation is correct]
	If $ \judgetype{\emptyset}{e}{T} $ and $ o : T $
	\[
	e \Downarrow_{B}^{D} o \; \text{if and only if} \; 
	e \Downarrow_{C}^{D} o
 	\]
\end{proposition}

\begin{corollary}[\lazyD{} hyper-coercion is correct]
	If $ \judgetype{\emptyset}{e}{T} $ and $ o : T $
	\[
	e \Downarrow^{D}_{B} o \; \text{if and only if} \; 
	e \Downarrow^{D}_{H} o
	\]
\end{corollary}

\subsection{$Lazy UD$ Hyper-coercion}

\section{Conclusion} \label{sec:conclude}

%% Acknowledgments
\begin{acks}                            %% acks environment is optional
                                        %% contents suppressed with 'anonymous'
  %% Commands \grantsponsor{<sponsorID>}{<name>}{<url>} and
  %% \grantnum[<url>]{<sponsorID>}{<number>} should be used to
  %% acknowledge financial support and will be used by metadata
  %% extraction tools.
  This material is based upon work supported by the
  \grantsponsor{GS100000001}{National Science
    Foundation}{http://dx.doi.org/10.13039/100000001} under Grant
  No.~\grantnum{GS100000001}{nnnnnnn} and Grant
  No.~\grantnum{GS100000001}{mmmmmmm}.  Any opinions, findings, and
  conclusions or recommendations expressed in this material are those
  of the author and do not necessarily reflect the views of the
  National Science Foundation.
\end{acks}


%% Bibliography
\bibliography{bibfile}


%% Appendix
\appendix
\section{Appendix}

Text of appendix \ldots

\end{document}
